\documentclass{article}
\usepackage[utf8]{inputenc}
\usepackage{listings}
\usepackage{xcolor}

\newcommand{\inlinecode}[1]{\colorbox{black}{\lstinline[basicstyle=\ttfamily\color{white}]|#1|}}
\newcommand{\varname}[1]{{\lstinline[basicstyle=\ttfamily\color{gray}]|#1|}}
\newcommand{\interface}[1]{{\lstinline[basicstyle=\ttfamily\color{brown}]|#1|}}
\newcommand{\ipaddress}[1]{{\lstinline[basicstyle=\ttfamily\color{purple}]|#1|}}
\newcommand{\hostname}[1]{{\lstinline[basicstyle=\ttfamily\color{teal}]|#1|}}
\newcommand{\path}[1]{{\lstinline[basicstyle=\ttfamily\color{blue}]|#1|}}


\author{Denys Vitali}
\title{Esperienza 9 - OpenVPN}
\begin{document}
\maketitle
\pagebreak
\section{Introduzione}
\subsection{Tipi di VPN}
\subsubsection{PPTP}
Metodo obsoleto di implementare delle VPN, con diversi problemi di sicurezza noti.
PPTP si basa sul protocollo TCP ed un protocollo di tunnelling (GRE) usato per incapsulare i pacchetti PPP\footnote{Point-to-Point Protocol}.
\\
La specifica PPTP non descrive funzionalità di cifratura o autenticazione, ma l'implementazione 
più comune (utilizzata nei prodotti di Microsoft Windows) implementa diversi livelli di sicurezza ed autenticazione. \\
\\
La porta utilizzata è la porta TCP/1723. Nell'implementazione di Microsoft il traffico PPP può essere
autenticato con PAP\footnote{Password Authentication Protocol}, CHAP\footnote{Challenge-Handshake Authentication Protocol} e MS-CHAP\footnote{Versione di Microsoft del CHAP} v1 o v2.

\subsubsection{IPsec}
L'Internet Protocol Security (IPsec) è un insieme di protocolli utilizzati per l'autenticazione e la cifratura di pacchetti inviati su una rete.\\
IPsec viene utilizzato per proteggere il flusso di dati fra due host (host-to-host), fra
una coppia di gateway (network-to-network) oppure fra un gateway ed un host (network-to-host).\\
IPsec usa una crittografia end-to-end per rendere sicura la connessione.
\\
L'insieme di protocolli include le seguenti funzionalità:
\begin{itemize}
    \item \textbf{Authentication Headers (AH)}\\
    Fornice garanzia sull'integrità dei dati e la loro origine per datagrammi IP, fornendo una protezione contro i replay attacks\footnote{Attacco nel quale il pacchetto cifrato viene ritrasmesso / ritardato}
    \item \textbf{Encapsulating Security Payloads (ESP)}\\
    Fornisce confidenzialità, data-origin authentication, integrità connectionless ed un servizio anti-replay.
    \item \textbf{Security Associations (SA)}\\
    Fornisce un'insieme di algoritmi e dati che forniscono i parametri necessari per le operazioni di AH e/o ESP.
\end{itemize}

\subsubsection{OpenVPN}
Software Open Source che implementa le tecniche di VPN per creare una connessione sicura point-to-point oppure site-to-site.
Implementa un protocollo di sicurezza personalizzato che usa SSL/TLS per lo scambio delle chiavi. \\
Il software è in grado di superare NATs e firewalls.\\
OpenVPN permette ai peers di autenticarsi a vicenda usando pre-shared keys, certificati oppure combinazioni di username / password. \\
In una configurazione multi-client il server può rilasciare un certificato di autenticazione per ogni client, usando delle firme ed una CA\footnote{Certificate Authority}.\\
Il programma utilizza molto le libreria di cifratura di OpenSSL, così come il protocollo TLS, ed include diverse funzionalità di sicurezza e di controllo.\\
È possibile utilizzare OpenVPN su diversi sistemi, inclusi quelli embedded (come OpenWRT o DD-WRT).\\
\\
La porta utilizzata da OpenVPN è solitamente la UDP/1194.

\pagebreak
\section{Creazione dell'ambiente per GNS3}

\subsection{Setup dei container}
Per questa esperienza useremo 3 container.
\begin{itemize}
    \item \hostname{Alpine-Client} (\ipaddress{192.168.0.2}, \interface{br0})
    \item \hostname{Alpine-VPN-Server} (\ipaddress{192.168.3.2}, \interface{br1})
    \item \hostname{pr-h1} (\ipaddress{10.100.0.3}, connesso ad \interface{eth1} di \hostname{Alpine-VPN-Server})
\end{itemize}

\subsection{Creazione delle interfacce di bridge}
\subsubsection{\interface{br0}}
\interface{br0} si occupa di gestire la sottorete \ipaddress{192.168.0.0/24},
configuriamola quindi come tale:
\begin{lstlisting}
    # brctl addbr br0
    # ip addr add 192.168.0.1/24 dev br0
    # ip link set br0 up
\end{lstlisting}

\subsubsection{\interface{br1}}
\interface{br0} si occupa di gestire la sottorete \ipaddress{192.168.3.0/24},
configuriamola quindi come tale:
\begin{lstlisting}
    # brctl addbr br1
    # ip addr add 192.168.3.1/24 dev br1
    # ip link set br1 up
\end{lstlisting}

\subsection{Configurazione di iptables per il routing dei pacchetti}
Una volta configurate le interfacce di bridge dobbiamo configurare iptables per 
fare il forwarding dei pacchetti provenienti da \interface{br0}, verso \interface{wlp7s0} (interfaccia wireless)
e viceversa.

\subsubsection{\interface{br0}}
\begin{lstlisting}
# iptables -A FORWARD --in-interface wlp7s0  \
    --out-interface br0 --destination 192.168.0.0/24 \
    -j ACCEPT

# iptables -A FORWARD --in-interface br0 \
    --out-interface wlp7s0 -j ACCEPT

# iptables -t nat -A POSTROUTING \ 
    --out-interface wlp7s0 \
    -j MASQUERADE
\end{lstlisting}
\vspace{1em}
Testiamo quindi il ping a \ipaddress{1.1.1.1} su \hostname{Alpine-Client} (\ipaddress{192.168.0.2}):
\begin{lstlisting}
Alpine-Client - / # ping 1.1.1.1
PING 1.1.1.1 (1.1.1.1): 56 data bytes
64 bytes from 1.1.1.1: seq=0 ttl=54 time=32.093 ms
64 bytes from 1.1.1.1: seq=1 ttl=54 time=41.078 ms
64 bytes from 1.1.1.1: seq=2 ttl=54 time=42.044 ms
^C
--- 1.1.1.1 ping statistics ---
3 packets transmitted, 3 packets received, 0% packet loss
round-trip min/avg/max = 32.093/38.405/42.044 ms
\end{lstlisting}
\vspace{1em}
\subsubsection{\interface{br1}}
\begin{lstlisting}
# iptables -A FORWARD --in-interface wlp7s0  \
    --out-interface br1 --destination 192.168.3.0/24 \
    -j ACCEPT

# iptables -A FORWARD --in-interface br1 \
    --out-interface wlp7s0 -j ACCEPT
\end{lstlisting}
\vspace{1em}
Proviamo ora il ping a \ipaddress{1.1.1.1} su \hostname{Alpine-VPN-Server} (\ipaddress{192.168.3.2}):
\begin{lstlisting}
Alpine-VPN-Server - / # ping 1.1.1.1
PING 1.1.1.1 (1.1.1.1): 56 data bytes
64 bytes from 1.1.1.1: seq=0 ttl=54 time=23.689 ms
64 bytes from 1.1.1.1: seq=1 ttl=54 time=77.014 ms
^C
--- 1.1.1.1 ping statistics ---
2 packets transmitted, 2 packets received, 0% packet loss
round-trip min/avg/max = 23.689/50.351/77.014 ms
\end{lstlisting}
\vspace{1em}
Infine testiamo e verifichiamo che le \hostname{Alpine-Client} ed \hostname{Alpine-VPN-Server} non si possano raggiungere a vicenda,
e che la sottorete privata \ipaddress{10.100.0.0/24} sia raggiungibile solo da \hostname{Alpine-VPN-Server}, simulando quindi il
caso in cui la sottorete \ipaddress{10.100.0.0/24} è privata e i due attori siano dietro ad un firewall.
Apriamo infine la porta UDP/1194 per \hostname{Alpine-VPN-Server} verso l'esterno, in modo che Alpine-Client possa connettersi.


\begin{lstlisting}
Alpine-Client - / # ping 192.168.0.1
PING 192.168.0.1 (192.168.0.1): 56 data bytes
64 bytes from 192.168.0.1: seq=0 ttl=64 time=0.115 ms
^C
--- 192.168.0.1 ping statistics ---
1 packets transmitted, 1 packets received, 0% packet loss
round-trip min/avg/max = 0.115/0.115/0.115 ms

Alpine-Client - / # ping 192.168.3.2
PING 192.168.3.2 (192.168.3.2): 56 data bytes
^C
--- 192.168.3.2 ping statistics ---
2 packets transmitted, 0 packets received, 100% packet loss
\end{lstlisting}

\begin{lstlisting}
Alpine-VPN-Server - / # ping 192.168.3.1
PING 192.168.3.1 (192.168.3.1): 56 data bytes
64 bytes from 192.168.3.1: seq=0 ttl=64 time=0.241 ms
^C
--- 192.168.3.1 ping statistics ---
1 packets transmitted, 1 packets received, 0% packet loss
round-trip min/avg/max = 0.241/0.241/0.241 ms

Alpine-VPN-Server - / # ping 192.168.0.2
PING 192.168.0.2 (192.168.0.2): 56 data bytes
^C
--- 192.168.0.2 ping statistics ---
2 packets transmitted, 0 packets received, 100% packet loss
\end{lstlisting}
\vspace{1em}

\subsection{Abilitazione della porta UDP/1194 per \hostname{Alpine-VPN-Server}}
Al fine di permettere ad Alpine-Client di connettersi con OpenVPN (una volta configurato) dobbiamo
fare in modo che i pacchetti dalla rete \ipaddress{192.168.0.0/24} con destinazione \ipaddress{192.168.3.2} sulla porta UDP/1194
vengano reindirizzati a br1. Per farlo usiamo iptables come segue:

\begin{lstlisting}
# iptables -t nat -I POSTROUTING -p udp --dport 1194 \
    --destination 192.168.3.2 -j MASQUERADE \
    --out-interface br1
\end{lstlisting}
\vspace{1em}
e settiamo la tabella di forwarding specifica per i pacchetti con destinazione \ipaddress{192.168.3.2:1194/UDP}
con sorgente \interface{br0} / \ipaddress{192.168.0.0/24}.

\begin{lstlisting}
# iptables -I FORWARD -p udp --dport 1194 \
    --in-interface br0 --out-interface br1 \ 
    --source 192.168.0.0/24 \
    --destination 192.168.3.2 \
    -j ACCEPT
\end{lstlisting}
\vspace{1em}
Testiamo infine la configurazione nel seguente modo:
\begin{enumerate}
    \item Su \hostname{Alpine-VPN-Server} usiamo \inlinecode{nc -v -u -l -p 1194} per ascoltare su porta UDP 1194
    e stampare a schermo i byte ricevuti.
    \item Su \hostname{Alpine-Client} usiamo \inlinecode{date \| nc -vu 192.168.3.2 1194} per inviare la data
    corrente come messaggio UDP a 192.168.3.2:1194.
\end{enumerate}

Se tutto funziona nel modo corretto e le porte sono aperte come specificato precedentemente,
su \hostname{Alpine-VPN-Server} si vedrà apparire un output simile al seguente:
\begin{lstlisting}
Alpine-VPN-Server - / # nc -v -u -l -p 1194
listening on [::]:1194 ...
connect to 192.168.3.2:1194 from [::ffff:192.168.3.1]:49122 ([::ffff:192.168.3.1]:49122)
Fri May 18 10:40:23 UTC 2018
\end{lstlisting}

Questo ci indica che la connessione via UDP da un qualsiasi client della sottorete 
\ipaddress{192.168.0.0/24} (ed interfaccia \varname{br0})
\pagebreak


\section{Installazione e configurazione di OpenVPN}
Il nostro obiettivo è quello di poter comunicare con la sottorete \ipaddress{10.100.0.0/24} da
\ipaddress{192.168.0.0/24} (\interface{br0}) utilizzando \ipaddress{192.168.3.2} (\interface{br1})
come server OpenVPN.

\subsection{Installazione del pacchetto OpenVPN}
Al fine di installare e configurare OpenVPN dobbiamo dapprima scaricare ed installare 
OpenVPN su \hostname{Alpine-VPN-Server}

\begin{lstlisting}
Alpine-VPN-Server - / # apk add openvpn
(1/8) Installing libelf (0.8.13-r3)
(2/8) Installing libmnl (1.0.4-r0)
(3/8) Installing jansson (2.10-r0)
(4/8) Installing libnftnl-libs (1.0.8-r1)
(5/8) Installing iptables (1.6.1-r1)
(6/8) Installing iproute2 (4.13.0-r0)
Executing iproute2-4.13.0-r0.post-install
(7/8) Installing lzo (2.10-r2)
(8/8) Installing openvpn (2.4.4-r1)
Executing openvpn-2.4.4-r1.pre-install
Executing busybox-1.27.2-r7.trigger
OK: 8 MiB in 19 packages
\end{lstlisting}


\subsection{Configurazione di OpenVPN}
Inseriamo ora la seguente configurazione in \path{/etc/openvpn/openvpn.conf}
\begin{lstlisting}
local "Public Ip address"
port 1194
proto udp
dev tun
ca openvpn_certs/server-ca.pem
cert openvpn_certs/server-cert.pem
dh openvpn_certs/dh1024.pem #to generate by hand #openssl dhparam -out dh1024.pem 1024
server 10.0.0.0 255.255.255.0
ifconfig-pool-persist ipp.txt
push "route 10.0.0.0 255.0.0.0"
push "dhcp-option DNS 10.0.0.1"
keepalive 10 120
comp-lzo
user nobody
group nobody
persist-key
persist-tun
status /var/log/openvpn-status.log
log-append  /var/log/openvpn.log
verb 3
\end{lstlisting}
\end{document}